%%%%%%%%%%%%%%%%%%%%%%%%%%%%%%%%%%%%%%%%%
% Awesome Resume/CV
% XeLaTeX Template
% Version 1.2 (27/3/2017)
%
% This template has been downloaded from:
% http://www.LaTeXTemplates.com
%
% Original author:
% Claud D. Park (posquit0.bj@gmail.com) with modifications by 
% Vel (vel@latextemplates.com)
%
% License:
% CC BY-NC-SA 3.0 (http://creativecommons.org/licenses/by-nc-sa/3.0/)
%
% Important note:
% This template must be compiled with XeLaTeX, the below lines will ensure this
%!TEX TS-program = xelatex
%!TEX encoding = UTF-8 Unicode
%
%%%%%%%%%%%%%%%%%%%%%%%%%%%%%%%%%%%%%%%%%

%----------------------------------------------------------------------------------------
%	PACKAGES AND OTHER DOCUMENT CONFIGURATIONS
%----------------------------------------------------------------------------------------

\documentclass[11pt, a4paper]{awesome-cv} % A4 paper size by default, use 'letterpaper' for US letter

\geometry{left=2cm, top=1.5cm, right=2cm, bottom=2cm, footskip=.5cm} % Configure page margins with geometry

\fontdir[fonts/] % Specify the location of the included fonts

% Color for highlights
\colorlet{awesome}{awesome-skyblue} % Default colors include: awesome-emerald, awesome-skyblue, awesome-red, awesome-pink, awesome-orange, awesome-nephritis, awesome-concrete, awesome-darknight
%\definecolor{awesome}{HTML}{CA63A8} % Uncomment if you would like to specify your own color

% Colors for text - uncomment and modify
%\definecolor{darktext}{HTML}{414141}
%\definecolor{text}{HTML}{414141}
%\definecolor{graytext}{HTML}{414141}
%\definecolor{lighttext}{HTML}{414141}

\renewcommand{\acvHeaderSocialSep}{\quad\textbar\quad} % If you would like to change the social information separator from a pipe (|) to something else
\usepackage{amsmath}
\newcommand{\sst}[1]{$ ^{\text{\tiny{#1}}} $} %If you would like to use superscript
\newcommand{\selfauthor}{{\bodyfont{Saikat Banerjee}}}

%----------------------------------------------------------------------------------------
%	PERSONAL INFORMATION
%	Comment any of the lines below if they are not required
%----------------------------------------------------------------------------------------

\name{Saikat}{Banerjee}
\address{T4/103, Am Fa{\ss}berg 11, G\"ottingen 37077, Germany}
\mobile{(+49) 17621103442}

\email{saikat.banerjee@mpibpc.mpg.de}
\homepage{saik.at}
\github{banskt}
\linkedin{banskt}
\dob{8 Sep. 1985}
\nationality{Indian}
%\skype{skypeid}
%\stackoverflow{SOid}{SOname}
%\twitter{@twit}
%\linkedin{linkedin name}
%\reddit{reddit account}
%\xing{xing name}
%\extrainfo{test} % Other text you want to include on this line
\photo[rectangle,edge,right]{3.2cm}{../images/sbanerjee_resume_2019_02_640x640.jpg}

\position{Statistical Genetics{\enskip\cdotp\enskip}Bayesian Methods{\enskip\cdotp\enskip}Systems Medicine{\enskip\cdotp\enskip}Data Science} % Your expertise/fields
\quote{Scientist / ex-entrepreneur / graphic designer. I write equations, develop software and lead projects to understand biomedical data. I solve problems, often learning new skills at a professional level.}

\makecvfooter{\today}{S.~Banerjee~~~·~~~Résumé}{\thepage/\pageref{LastPage}} % Specify the letter footer with 3 arguments: (<left>, <center>, <right>), leave any of these blank if they are not needed

%----------------------------------------------------------------------------------------

\begin{document}

\makecvheader % Print the header

%----------------------------------------------------------------------------------------
%	CV/RESUME CONTENT
%	Each section is imported separately, open each file in turn to modify content
%----------------------------------------------------------------------------------------

%----------------------------------------------------------------------------------------
%	SECTION TITLE
%----------------------------------------------------------------------------------------

\cvsection{Experience}

%----------------------------------------------------------------------------------------
%	SECTION CONTENT
%----------------------------------------------------------------------------------------

\begin{cventries}

%------------------------------------------------

\cventry
{Advisor: Dr. Johannes S{\"o}ding} % Job title
{Postdoctoral Fellow, Max Planck Institute for Biophysical Chemistry} % Organization
{G{\"o}ttingen, Germany} % Location
{Jun. 2015 -- Present} % Date(s)
{ % Description(s) of tasks/responsibilities
\begin{cvitems}
\item {Developed three statistical methods for understanding disease mechanism from large scale biomedical data.
       Technical details: (1) Bayesian multiple logistic regression for finemapping in post-GWAS analysis, (2) large-scale trans-eQTL discovery from RNA-Seq data, (3) Bayesian multi-omics approach to combine GWAS and eQTL data.}
\item {Communicated and collaborated with medical doctors leading to a first author publication.}
\item {Presented our work at an international conference (ISMB 2019). Invited to speak at the University of G{\"o}ttingen.}
\item {Supervised a Master's thesis and mentored three internship students.}
%\item {Method development and analyses of medical data to understand the genetic origin of common, non-infectious diseases.}
%\item {Public speaking for professional audience, teaching and mentoring for students.}
%\item {B-LORE: a software for post-GWAS analyses using Bayesian multiple logistic regression with variable selection.}
%\item {TEJAAS: a high-throughput method for detecting trans-eQTLs.}
\end{cvitems}
}

%------------------------------------------------    

\cventry
{Advisor: Prof. Biman Bagchi} % Job title
{Research Associate, Indian Institute of Science} % Organization
{Bengaluru, India} % Location
{Aug. 2014 - May 2015} % Date(s)
{ % Description(s) of tasks/responsibilities
\begin{cvitems}
\item {Led two projects with challenging deadlines: (1) Understanding the origin of long-range hydrophobic force, and (2) Role of biological water in the hydration shell of proteins.}
\item {Communicated both of them in international peer-reviewed journals, one of them as first author.}
\item {In charge of administration and maintenance of high performance computing (HPC) cluster from Aug. 2011 to May 2015.}
\end{cvitems}
}

%------------------------------------------------

\cventry
{A B2B company for word-of-mouth marketing} % Job title
{Co-founder, Beejig} % Organization
{Bengaluru, India} % Location
{Jan. 2013 - Feb. 2014} % Date(s)
{ % Description(s) of tasks/responsibilities
\begin{cvitems}
\item {Extensively involved in conceptualization, market research and app development.}
\item {Raised initial funding of \$20000 and recruited sales manager along with a team of two sales executives.}
\item {Deployed our marketing tool for Lakme India chain of salons in Bengaluru. Their sales increased by 12\% in only two months, and we were invited to launch our program in Mumbai.}
\item {Organized the dissolution of our company by selling the assets to a bigger company.}
\end{cvitems}
}

%------------------------------------------------

\end{cventries}

%----------------------------------------------------------------------------------------
%	SECTION TITLE
%----------------------------------------------------------------------------------------

\cvsection{Education}

%----------------------------------------------------------------------------------------
%	SECTION CONTENT
%----------------------------------------------------------------------------------------

\begin{cventries}

%------------------------------------------------

\eduentry
{} % Degree
{PhD (Computational Biophysics), Indian Institute of Science} % Institution
{2014} % Location
{} % Date(s)
{ % Description(s) bullet points
More than 10 peer-reviewed publications.
}

\eduentry
{} % Degree
{M.S. (Chemistry), Indian Institute of Science} % Institution
{2009} % Location
{} % Date(s)
{ % Description(s) bullet points
Class rank 2\sst{nd} with a CGPA of 6.9 out of 8.
}

\eduentry
{} % Degree
{B.Sc., University of Calcutta} % Institution
{2007} % Location
{} % Date(s)
{ % Description(s) bullet points
Ranked 1\sst{st} among all colleges in the university.
}

%------------------------------------------------

\end{cventries}

%----------------------------------------------------------------------------------------
%	SECTION TITLE
%----------------------------------------------------------------------------------------

\cvsection{Skills}

%----------------------------------------------------------------------------------------
%	SECTION CONTENT
%----------------------------------------------------------------------------------------

\begin{cvskills}

%------------------------------------------------

\cvskill
{Programming} % Category
{Python, FORTRAN, C++, Java} % Skills

%------------------------------------------------

\cvskill
{Bioinformatics}
{GWAS, EQTL, Finemapping, PrediXcan, omics data}

%------------------------------------------------

\cvskill
{Biophysics} % Category
{LAMMPS, GROMACS, VMD} % Skills

%------------------------------------------------

\cvskill
{HPC}
{Slurm, LSF, SGE}

%------------------------------------------------

\cvskill
{Web} % Category
{HTML5, CSS, PHP, Node.JS} % Skills

%------------------------------------------------

\cvskill
{Others} % Category
{Linux, Bash, Git, {\LaTeX}, Adobe Illustrator, Adobe Photoshop, Inkscape} % Skills

%------------------------------------------------

\cvskill
{Languages} % Category
{Bengali (native), English (native), Hindi (native), German (beginner)} % Skills

%------------------------------------------------

\end{cvskills}


\newpage % Force a new page for looks

%----------------------------------------------------------------------------------------
%	SECTION TITLE
%----------------------------------------------------------------------------------------

\cvsection{Software}

%----------------------------------------------------------------------------------------
%	SECTION CONTENT
%----------------------------------------------------------------------------------------

\begin{cvskills}

%------------------------------------------------

\cvskill
{B-LORE} % Category
{Bayesian multiple logistic regression with variable selection.} % Skills

%------------------------------------------------

\cvskill
{TEJAAS} % Category
{L$_\text{2}$ regularized `reverse' multiple linear regression for discovering trans-eQTLs.} % Skills

%------------------------------------------------

\end{cvskills}

%----------------------------------------------------------------------------------------
%	SECTION TITLE
%----------------------------------------------------------------------------------------

\cvsection{Select Publications}

%----------------------------------------------------------------------------------------
%	SECTION CONTENT
%----------------------------------------------------------------------------------------

\begin{cvpubs}

\pubitem{1}%
        {Bayesian multiple logistic regression for case-control GWAS}
        {\selfauthor{}, Lingyao Zeng, Heribert Schunkert and Johannes S\"oding}
        {{\slshape PLOS Genetics}, DOI:10.1371/journal.pgen.1007856 (2018)}
        {https://doi.org/10.1371/journal.pgen.1007856}

\pubitem{3}%
        {Orientational order as the origin of the long-range hydrophobic effect}
        {\selfauthor{}, Rakesh S. Singh and Biman Bagchi}
        {{\slshape The Journal of Chemical Physics}, {\bfseries 142}, 134505 (2015)}
        {https://doi.org/10.1063/1.4916744}

\pubitem{7}%
        {Diffusion on a rugged energy landscape with spatial correlation}
        {\selfauthor{}, Rajib Biswas, Kazuhiko Seki and Biman Bagchi}
        {{\slshape The Journal of Chemical Physics}, {\bfseries 141}, 124105 (2014)}
        {https://doi.org/10.1063/1.4895905}


\pubitem{9}%
        {Fluctuating micro-heterogeneity in water--tert-butyl alcohol mixtures and
        lambda-type divergence of the mean cluster size with phase transition-like
        multiple anomalies}
        {\selfauthor{}, Jonathan Furtado and Biman Bagchi}
        {{\slshape The Journal of Chemical Physics}, {\bfseries 140}, 194502 (2014) {\bfseries [Featured Article]}}
        {https://doi.org/10.1063/1.4874637}

\pubitem{10}%
        {Structural transformations, composition anomalies and a dramatic
        collapse of linear polymer chains in dilute ethanol--water mixtures}
        {\selfauthor{}, Rikhia Ghosh, and Biman Bagchi}
        {{\slshape The Journal of Physical Chemistry~B}, {\bfseries 116}, 3713--3722 (2012)}
        {https://doi.org/10.1021/jp2085439}

\pubitem{12}%
        {Theoretical and computational analysis of static and dynamic anomalies
        in water--DMSO binary mixture at low DMSO concentrations}
        {Susmita Roy, \selfauthor{} and Biman Bagchi}
        {{\slshape The Journal of Physical Chemistry~B}, {\bfseries 115}, 685--692 (2011)}
        {https://doi.org/10.1021/jp109622h}

\pubitem{13}%
        {Enhanced pair hydrophobicity in the water--dimethyl sulfoxide (DMSO)
        binary mixture at low DMSO concentrations}
        {\selfauthor{}, Susmita Roy and Biman Bagchi}
        {{\slshape The Journal of Physical Chemistry~B}, {\bfseries 114}, 12875--12882 (2010)}
        {https://doi.org/10.1021/jp1045645}

\end{cvpubs}

%----------------------------------------------------------------------------------------
%	SECTION TITLE
%----------------------------------------------------------------------------------------

\cvsection{Select Presentations}

%----------------------------------------------------------------------------------------
%	SECTION CONTENT
%----------------------------------------------------------------------------------------

\begin{cventries}

%------------------------------------------------

\cventry
{Bayesian logistic regression for case-control GWAS} % Role
{Annual meeting of the International Society for Molecular Biology} % Event
{Basel, Switzerland} % Location
{Jul. 2019} % Date(s)
{ % Description(s)
}

%------------------------------------------------

\cventry
{Invited talk at University of G{\"o}ttingen} % Role
{Advanced seminar for statistical genetics} % Event
{G{\"o}ttingen, Germany} % Location
{Aug. 2017} % Date(s)
{ % Description(s)
}

%------------------------------------------------

\end{cventries}

%----------------------------------------------------------------------------------------
%	SECTION TITLE
%----------------------------------------------------------------------------------------

\cvsection{Honors \& Awards}

\begin{cvhonors}

%------------------------------------------------

\cvhonor
{Best Poster} % Award
{Faraday Discussions, Royal Society of Chemistry} % Event
{} % Location
{2017} % Date(s)

%------------------------------------------------

\cvhonor
{Best Poster} % Award
{Frontier Meeting in Chemical Biology} % Event
{} % Location
{2009} % Date(s)

%------------------------------------------------

\cvhonor
{Gold Medalist} % Award
{Secured 1\sst{st} position at B.Sc. University of Calcutta} % Event
{} % Location
{2007} % Date(s)

%------------------------------------------------

\cvhonor
{National Merit Scholarship} % Award
{Sponsored by Govt. of India after 10\sst{th} standard.} % Event
{} % Location
{2002} % Date(s)

%------------------------------------------------

\end{cvhonors}

%----------------------------------------------------------------------------------------
%	SECTION TITLE
%----------------------------------------------------------------------------------------

\cvsection{Extracurricular Activity}

%----------------------------------------------------------------------------------------
%	SECTION CONTENT
%----------------------------------------------------------------------------------------

\begin{cventries}

%------------------------------------------------

\cventry
{} % Job title
{Graphic design and web development, Freelancer} % Organization
{} % Location
{} % Date(s)
{ % Description(s) of tasks/responsibilities
\begin{cvitems}
\item {Curated award-winning logos and created web / brand identity for more than 20 startups.}
\item {Consulted the design and development of the iOS app `Isle of Miles'.}
\item {Designed two book covers for Oxford University Press.}
\end{cvitems}
}

%------------------------------------------------

\cventry
{} % Job title
{Hobbies} % Organization
{} % Location
{} % Date(s)
{ % Description(s) of tasks/responsibilities
\begin{cvitems}
\item {Photography, Hiking, Long distance biking}
\end{cvitems}
}

\end{cventries}

%%----------------------------------------------------------------------------------------
%	SECTION TITLE
%----------------------------------------------------------------------------------------

\cvsection{Writing}

%----------------------------------------------------------------------------------------
%	SECTION CONTENT
%----------------------------------------------------------------------------------------

\begin{cventries}

%------------------------------------------------

\cventry
{Founder \& Writer} % Role
{A Guide for Developers in Start-up} % Title
{Facebook Page} % Location
{Jan. 2015 - PRESENT} % Date(s)
{ % Description(s)
\begin{cvitems}
\item {Drafted daily news for developers in Korea about IT technologies, issues about start-up.}
\end{cvitems}
}

%------------------------------------------------

\cventry
{Undergraduate Student Reporter} % Role
{AhnLab} % Title
{S.Korea} % Location
{Oct. 2012 - Jul. 2013} % Date(s)
{ % Description(s)
\begin{cvitems}
\item {Drafted reports about IT trends and Security issues on AhnLab Company magazine.}
\end{cvitems}
}

%------------------------------------------------

\end{cventries}
%%----------------------------------------------------------------------------------------
%	SECTION TITLE
%----------------------------------------------------------------------------------------

\cvsection{Program Committees}

%----------------------------------------------------------------------------------------
%	SECTION CONTENT
%----------------------------------------------------------------------------------------

\begin{cvhonors}

%------------------------------------------------

\cvhonor
{Organizer \& Co-director} % Position
{1st POSTECH Hackathon} % Committee
{S.Korea} % Location
{2013} % Date(s)
    
%------------------------------------------------

\cvhonor
{Staff} % Position
{7th Hacking Camp} % Committee
{S.Korea} % Location
{2012} % Date(s)

%------------------------------------------------

\cvhonor
{Problem Writer} % Position
{1st Hoseo University Teenager Hacking Competition} % Committee
{S.Korea} % Location
{2012} % Date(s)

%------------------------------------------------

\cvhonor
{Staff \& Problem Writer} % Position
{JFF(Just for Fun) Hacking Competition} % Committee
{S.Korea} % Location
{2012} % Date(s)

%------------------------------------------------

\end{cvhonors}

%----------------------------------------------------------------------------------------

\end{document}
